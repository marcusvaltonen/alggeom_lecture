% !TEX encoding = UTF-8 Unicode

\documentclass[11pt,a4paper]{article}

\usepackage[T1]{fontenc}
\usepackage[utf8]{inputenc}

\usepackage[sc]{mathpazo}
\linespread{1.05} % Line spacing - Palatino needs more space between lines
\usepackage{microtype} % Slightly tweak font spacing for aesthetics

\usepackage{geometry}
\usepackage[pdftex]{graphicx}

\usepackage{amsmath}
\usepackage{amssymb}
\usepackage{amsthm}

\usepackage{epstopdf}

\usepackage[font=footnotesize,labelfont=bf]{caption}
\usepackage[font=footnotesize,labelfont=bf]{subcaption}
\usepackage{hyphenat}

\usepackage{mathrsfs}
\newtheorem{theorem}{Theorem}
\newtheorem{corollary}{Corollary}
\newtheorem{prop}{Proposition}

% Math in header should be boldface
\makeatletter
\g@addto@macro\bfseries{\boldmath}
\makeatother

% My own commands
\usepackage{gensymb} % degree sign
\renewcommand{\epsilon}{\varepsilon}
\renewcommand{\phi}{\varphi}

\newcommand{\fr}[2]{\frac{#1}{#2}}
\newcommand{\pf}[2]{\dfrac{\partial #1}{\partial #2}}
\newcommand{\cd}{\cdot}
% \newcommand{\T}{\!\mathsf{T}}
\newcommand{\T}{T}
\newcommand{\te}[1]{\text{#1}}
\newcommand{\id}{\mathrm{d}}
\usepackage{bm}
\newcommand{\mat}[1]{\bm{#1}}
\newcommand{\nullspace}[1]{\mathcal{N}{#1}}
\newcommand{\lagrange}[1]{\mathcal{L}{#1}}
\usepackage{mathtools}

\usepackage{bbm}
\newcommand{\N}{\ensuremath{\mathbbm{N}}}
\newcommand{\Z}{\ensuremath{\mathbbm{Z}}}
\newcommand{\Q}{\ensuremath{\mathbbm{Q}}}
\newcommand{\R}{\ensuremath{\mathbbm{R}}}
\newcommand{\C}{\ensuremath{\mathbbm{C}}}

\newcommand{\lr}{\ensuremath{\mathrm{\quad\Leftrightarrow\quad}}}
\newcommand{\ra}{\ensuremath{\mathrm{\quad\Rightarrow\quad}}}
\newcommand{\lra}{\ensuremath{\mathrm{\longrightarrow}}}
\newcommand{\f}[3]{#1\,:\,#2\,\lra\,#3}
\newcommand{\norm}[1]{\left\|#1\right\|}
\DeclareMathOperator{\sgn}{sgn}
\DeclareMathOperator{\diag}{diag}
\DeclareMathOperator{\atan}{atan2}
\DeclareMathOperator{\tr}{tr}
\DeclareMathOperator{\adj}{adj}
\DeclareMathOperator*{\argmax}{argmax}
\DeclareMathOperator{\nulldim}{nulldim}
\DeclareMathOperator{\rank}{rank}
\DeclareMathOperator{\lt}{LT}
\newcommand{\Rop}[1]{\mathcal{R}_{#1}}

\makeatletter
\DeclareRobustCommand\eg{\emph{e.g}\@ifnextchar.{}{.\@}}
\DeclareRobustCommand\etal{\emph{et~al}\@ifnextchar.{}{.\@}}
\DeclareRobustCommand\ie{\emph{i.e}\@ifnextchar.{}{.\@}}
\DeclareRobustCommand\cf{\emph{cf}\@ifnextchar.{}{.\@}}
\DeclareRobustCommand\NB{\emph{N.B}\@ifnextchar.{}{.\@}}
\DeclareRobustCommand\wrt{w.r.t\@ifnextchar.{}{.\@}}
\makeatother

\usepackage{hyperref}


%----------------------------------------------------------------------------------------
%   TITLE SECTION
%----------------------------------------------------------------------------------------

\title{
\normalfont \normalsize
\textsc{Lund University --- Algebraic Geometry} \\ [7pt]
\Large Practical aspects of solving a system of polynomial equations \\
}
\author{Marcus Valtonen \"{O}rnhag} % Your name

\date{\normalsize\today} % Today's date or a custom date

\begin{document}

\maketitle % Print the title

\section{Introduction}
In this lecture we will discuss how to optimize the computations, both for speed and
accuracy, given a polynomial system of equations. In the first part we will briefly go
throught the theory, and recommend further reading for self-study. Then, we will show
how to apply this theory and how it affects the solver. In the latter part we will use the
automatic solver by Larsson~\etal{}, which you should install and get accustomed to prior
to the lecture. The solver is available at~\url{http://people.inf.ethz.ch/vlarsson/misc/autogen_v0_5.zip}, and requires Macaulay2 to be installed. If you want to use C++ (which greatly decreases the computation time) you will need to install the Eigen library as well.

\section{Theory}
\subsection{Recapitulation of the course contents}
Using multi-index notation,
let~$\mat{x^{\alpha}}$ denote a \emph{monomial} of degree $|\mat{\alpha}|$.
Our main goal is to solve a polynomial system of equations
\begin{equation}\label{kappa:eq:polsys}
    \begin{aligned}
    f_1(\mat{x}) & = 0, \\
                 & \hspace{0.575em} \vdots \\
    f_s(\mat{x}) & = 0,
    \end{aligned}
\end{equation}
where each polynomial equation can be expressed
as~$f=\sum_{\mat{\alpha}}c_{\mat{\alpha}}\mat{x^\alpha}$.
The set of all solutions to~\eqref{kappa:eq:polsys} is called an \emph{affine variety}, and is denoted
$\mat{V}(f_1,\ldots,f_s)\subset\C$.
Let~$\C[\mat{x}]$ denote the set
of polynomials in~$\mat{x}$ with coefficients in $\C$.
An \emph{ideal} $I\subset\C[\mat{x}]$ is an additive group satisfying the \emph{absorption property},~\ie{}~if~$f\in I$
and $h\in\C[\mat{x}]$ then $hf\in I$. Moreover, given a set of polynomials $f_1,\ldots,f_s\in\C[\mat{x}]$
we consider the ideal
\begin{equation}
\langle f_1,\ldots,f_s\rangle = \left\{\sum_{i=1}^sh_if_i\;:\;h_1,\ldots,h_s\in\C[\mat{x}]\right\},
\end{equation}
which we will refer to as the \emph{ideal generated by} $f_1,\ldots,f_s$.
The \emph{Hilbert Basis Theorem} states that every ideal of $\C[\mat{x}]$ is
finitely generated, \ie{} given an ideal~$I\subset\C[\mat{x}]$ there exist
$f_1,\ldots,f_s\in\C[\mat{x}]$ such that $I=\langle f_1,\ldots,f_s\rangle$.
Thus, the polynomial system of equations~\eqref{kappa:eq:polsys} is defined by the
generated ideal; however, the basis is in general not unique.

To be able to represent the system of equations uniquely, one must impose
a \emph{monomial ordering}, \eg{} the lexicographic order,
the graded lex order or the graded reverse lex order. Regardless of which monomial ordering
is chosen the \emph{leading term} (\wrt{}~the monomial ordering) is uniquely defined, and we shall
denote it $\lt(f)$. Furthermore, for an ideal $I$ define $\lt(I)\coloneqq\{\lt(f)\;:\;f\in I\}$. Then
a finite subset \mbox{$\mat{G}=\{g_1,\ldots,g_t\}\subset I$} is a \emph{Gröbner basis}
if $\langle\lt(g_1),\ldots,\lt(g_t)\rangle=\lt(I)$.

After this exposition one may ask: how does this relate to solving polynomial systems
of equations? The answer lies in the fact that we have efficient ways of computing
a Gröbner basis, by means of \emph{Buchberger's algorithm}.
For more details regarding the algorithm, and to get a deeper understanding of the subject,
the work of Cox~\etal{}~\cite{cox,cox2} is highly recommended.

\subsection{The action matrix method}\label{kappa:sec:actionmatrix}
Let us return to the polynomial system of equations~\eqref{kappa:eq:polsys}. Under the
assumption that the system has finitely many solutions, \ie{} when $\mat{V}(I)$ is finite,
it follows by the Finiteness Theorem (see \eg{}~\cite{cox2}) that $I$ is
zero-dimensional and the quotient space~\mbox{$A=\C[\mat{x}]/I$} is
finite dimensional. Given the \emph{coset}~$[f] = \{f+h\;:\;h\in I\}$,
consider the operator~\mbox{$T_f\::\:A\longrightarrow A$}, defined by~$T_f([g])=[fg]$.
It is easily seen that~$T_f$ is
linear with the property that~$T_f=T_g$ if and only if~$f-g\in I$.
Since the quotient space~$A$ is finite-dimensional this operation can be
represented by a matrix~$\mat{M}_f$, which is known as the \emph{action matrix}.
Furthermore, we may select a basis for~$A$,~\eg{} a monomial
basis~$\mathcal{B}=\{[\mat{x}^{\alpha_j}]\}_{j\in J}$,
typically obtained by (an improved version of)
Buchberger's algorithm.
When the action matrix~$\mat{M}_f=(m_{ij})$ acts on the basis elements we obtain a
linear combination of the monomials forming the basis, namely
\begin{equation}
    T_f([\mat{x}^{\alpha_j}]) = [f\mat{x}^{\alpha_j}]=\sum_{i\in J}m_{ij}[\mat{x}^{\alpha_i}]\;.
\end{equation}
This implies that,
% for some $h\in I$
% \begin{equation}
%     f\mat{x}^{\alpha_j}=\sum_{i\in J}m_{ij}\mat{x}^{\alpha_i},
% \end{equation}
% and
for $\mat{x}\in V(I)$,
\begin{equation}\label{kappa:eq:action}
f(\mat{x})\mat{x}^{\alpha_j}=\sum_{i\in J}m_{ij}\mat{x}^{\alpha_i}\;.
\end{equation}
By representing the basis~$\mathcal{B}$ with a vector $\mat{b}$, and using the fact
that~\eqref{kappa:eq:action} must hold for all basis elements, the problem can be reduced to
\begin{equation}
f(\mat{x})\mat{b}(\mat{x}) = \mat{M}_f^{\T}\mat{b}(\mat{x}),
\end{equation}
which we recognise as an eigenvalue problem.
Let us recapitulate: given a polynomial
system of equations, use Buchberger's algorithm to obtain a Gröbner basis. Create the
action matrix and compute the eigenvalues and eigenvectors to extract the solutions.

\subsection{Syzygy Modules}




\subsection{Saturation of an ideal}
\subsection{The Hidden variable trick}
\subsection{Eliminating variables}
\subsection{Beyond Gröbner bases}

\section{Applications}
\subsection{Syzygies}
\subsection{Saturation of an ideal}
Comparison to \emph{Rabinowitsch trick}.
\subsection{The hidden variable trick}
\subsection{Eliminating variables}
\subsection{Beyond Gröbner bases}

\bibliographystyle{apalike}
{\small
\bibliography{alggeom_lecture_marcus}}

 \end{document}
