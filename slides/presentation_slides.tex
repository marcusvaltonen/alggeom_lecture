\documentclass[aspectratio=169]{beamer}
\usepackage{animate}

% Show notes
\usepackage{pgfpages}
%\setbeameroption{show notes on second screen}

\usepackage[utf8]{inputenc}

\setbeamertemplate{note page}
{%
  % Header box
  \insertvrule{.27\paperheight}{white!90!black}
  \vskip-0.286\paperheight
  \begin{minipage}{0.4\textwidth}
    \insertslideintonotes{0.25}%
  \end{minipage}
  \begin{minipage}{0.55\textwidth}
    {\textbf \insertshortframetitle}
  \end{minipage}

  % Title
  \begin{minipage}{0.88\paperwidth}
  \insertnote
  \end{minipage}
}

\newcommand{\timenote}[3]{
    \note{
        \vskip-0.1\paperheight
        \hspace{4.44cm}Time: #1 -- #2 min
        \vskip0.086\paperheight
        {\scriptsize #3}}
}

\usepackage{mathtools}


\usepackage[]{algorithm2e}
\usepackage{bm}
\renewcommand{\vec}[1]{%
  \ifcat\noexpand#1\relax % check if the argument is a control sequence
    \bm{#1} % probably Greek
  \else
    \mathbf{#1} % single character
  \fi
}
\usepackage{amssymb}
\usepackage{amstext}
\usepackage{amsmath}
\usepackage{amsthm}
\usepackage{changepage}
\usepackage{mathrsfs}  % Used for mathscr 'L'
\usepackage{gensymb}  % degree sign
\usepackage{bm}
\usepackage{bbm}
\renewcommand{\epsilon}{\varepsilon}
\usepackage{cancel}
%\usepackage{breqn}
%\usepackage{enumitem}
\usepackage{booktabs}
%\usepackage[table]{xcolor}
\usepackage{multirow}

\usepackage{empheq}
%\usepackage[most]{tcolorbox}
%\newtcbox{\mymath}[1][]{%
%    nobeforeafter, math upper, tcbox raise base,
%    enhanced, colframe=blue!30!black,
%    colback=blue!30, boxrule=1pt,
%    #1}
%\tcbset{
%    enhanced,
%    breakable,
%    attach boxed title to top center,
%    top=1mm,
%    coltitle=black,
%    beforeafter skip=\baselineskip,
%}

\newcommand{\etal}{\emph{et~al.}}

% Graphics
\graphicspath{{../images/}{../matlab/}{../../camera_ready/images/}}
\DeclareGraphicsExtensions{.pdf,.jpeg,.png,.eps}

\newcommand{\FULLSCREEN}{0.65\textwidth}

\makeatletter
\newcommand{\vast}{\bBigg@{4}}
\newcommand{\Vast}{\bBigg@{5}}
\newcommand{\VAST}{\bBigg@{16}}
\newcommand{\VAst}{\bBigg@{14}}
\makeatother

\usepackage[export]{adjustbox}[2011/08/13]

\newcommand{\cammat}[2]{\left[#1\;|\;#2\right]}
\newcommand{\R}{\ensuremath{\mathbbm{R}}}
\newcommand{\K}{\ensuremath{\mathbbm{K}}}
\newcommand{\C}{\ensuremath{\mathbbm{C}}}
\renewcommand{\phi}{\varphi}
\DeclareMathOperator*{\argmax}{arg\,max}
\DeclareMathOperator*{\argmin}{arg\,min}
\DeclareMathOperator{\nulldim}{nulldim}
\DeclareMathOperator{\adj}{adj}
\newcommand{\mat}[1]{\bm{#1}}
\newcommand{\nullspace}[1]{\mathcal{N}{#1}}
\newcommand{\lr}{\ensuremath{\mathrm{\quad\Leftrightarrow\quad}}}
\newcommand{\A}{\ensuremath{\mathcal{A}}}
\newcommand{\N}{\ensuremath{\mathcal{N}}}
\newcommand{\reg}{\ensuremath{\mathcal{R}}}
\newcommand{\Real}{\ensuremath{\mathbbm{R}}}
\newcommand{\Z}{\ensuremath{\mathbbm{Z}}}
\newcommand{\Tp}{T}
\newcommand{\norm}[1]{\|#1\|}
\newcommand{\frobnorm}[1]{\norm{#1}_F}
\newcommand{\fr}[2]{\frac{#1}{#2}}
\newcommand{\pf}[2]{\frac{\partial #1}{\partial #2}}
\DeclareMathOperator{\tr}{tr}
\DeclareMathOperator{\diag}{diag}
\DeclareMathOperator{\atan}{atan2}
\DeclareMathOperator{\rank}{rank}

\newcommand{\T}{T}
\usepackage{setspace}
%\usepackage{xcolor}
\usepackage{tikz}
\usepackage{pgfplots}

%\usepackage{caption}
\usepackage{subcaption}

\def\thicknesslineplot{thick}
\def\thicknesshack{0.085cm}
\def\mythickness{ultra thick}
\def\thicknesslineplotTwo{thick}
\def\mythicknesssmall{thick}
\def\thicknesslineplotThree{very thick}
\def\boxplotheight{8.5cm}
\def\boxplotwidth{7.5cm}
\def\boxplotheightreal{3.5cm}

\definecolor{gtcolor}{rgb}{0,0,0}
\definecolor{ourcolor}{rgb}{0,0.4,0.6}
\definecolor{fitzgibboncolor}{rgb}{0,0.7,0.7}
\definecolor{dingcolor}{rgb}{1,0.2,0.3}
\definecolor{kukelovacolor}{rgb}{0,0.6,0}
\definecolor{bougnouxcolor}{rgb}{0,0.7,0.7}
\definecolor{valtonenoernhagcolor}{rgb}{0.7,0,0.7}
\definecolor{inliercolor}{rgb}{0.2,1,0.1}
\definecolor{outliercolor}{rgb}{1,0,0}
\definecolor{smallcolor1}{rgb}{0.,0,0}
\definecolor{smallcolor2}{rgb}{0,0,0}
\definecolor{hiddencolor}{rgb}{0.1,0.9,0.1}
\definecolor{elimcolor}{rgb}{0.6,0.1,0.6}

\makeatletter
\DeclareRobustCommand\eg{\emph{e.g}\@ifnextchar.{}{.\@}}
\DeclareRobustCommand\etal{\emph{et~al}\@ifnextchar.{}{.\@}}
\DeclareRobustCommand\ie{\emph{i.e}\@ifnextchar.{}{.\@}}
\DeclareRobustCommand\cf{\emph{cf}\@ifnextchar.{}{.\@}}
\DeclareRobustCommand\NB{\emph{N.B}\@ifnextchar.{}{.\@}}
\DeclareRobustCommand\wrt{w.r.t\@ifnextchar.{}{.\@}}
\makeatother

\newcommand{\CE}{\ensuremath{\mat{C}_\mathcal{E}}}
\newcommand{\CR}{\ensuremath{\mat{C}_\mathcal{R}}}
\newcommand{\CB}{\ensuremath{\mat{C}_\mathcal{B}}}
\newcommand{\CEk}[1]{\ensuremath{\mat{C}_{\mathcal{E}_{#1}}}}
\newcommand{\CRk}[1]{\ensuremath{\mat{C}_{\mathcal{R}_{#1}}}}
\newcommand{\CBk}[1]{\ensuremath{\mat{C}_{\mathcal{B}_{#1}}}}
\newcommand{\XE}{\ensuremath{\mat{X}_\mathcal{E}}}
\newcommand{\XR}{\ensuremath{\mat{X}_\mathcal{R}}}
\newcommand{\XB}{\ensuremath{\mat{X}_\mathcal{B}}}

\newcommand{\sing}[1]{\vec{\sigma}{#1}}
\usepackage{standalone}
\pgfplotsset{
    table/search path={../../graphs},
}

\makeatletter
\@namedef{ver@everyshi.sty}{} %Prevents some error caused by tikz, has to be agter xcolor
\makeatother
\usepackage{tikz}  % Used for problem geometry picture
\usepackage{pgfplots}
\usepackage{tikzpagenodes}
\usetikzlibrary{positioning}
\usetikzlibrary{arrows,shapes}
\usetikzlibrary{arrows.meta}
\tikzset{%
  thick arrow/.style={
     -{Triangle[angle=90:2pt 1.5]},
     line width=4mm,
     draw=gray,
     draw opacity=0.7,
  },
}
\pgfplotsset{compat=1.15}
\usetikzlibrary{shapes.misc, positioning}
\usetikzlibrary{arrows}
\usetikzlibrary{calc}

\pgfdeclarelayer{foreground}  % Used for correct opacity
\pgfdeclarelayer{background}  % Used for correct opacity
\pgfsetlayers{background,main,foreground}

\DeclareGraphicsRule{.tif}{png}{.png}{`convert #1 `dirname #1`/`basename #1 .tif`.png}

\usepackage{environ}  % Used for scaletikzpicturewidth
\makeatletter
\newsavebox{\measure@tikzpicture}
\NewEnviron{scaletikzpicturetowidth}[1]{%
  \def\tikz@width{#1}%
  \def\tikzscale{1}\begin{lrbox}{\measure@tikzpicture}%
  \BODY
  \end{lrbox}%
  \pgfmathparse{#1/\wd\measure@tikzpicture}%
  \edef\tikzscale{\pgfmathresult}%
  \BODY
}
\makeatother

\tikzset{add reference/.style={insert path={%
coordinate [pos=0,xshift=-0.5\pgflinewidth,yshift=-0.5\pgflinewidth] (#1 south west)
coordinate [pos=1,xshift=0.5\pgflinewidth,yshift=0.5\pgflinewidth]   (#1 north east)
coordinate [pos=.5] (#1 center)
(#1 south west |- #1 north east)     coordinate (#1 north west)
(#1 center     |- #1 north east)     coordinate (#1 north)
(#1 center     |- #1 south west)     coordinate (#1 south)
(#1 south west -| #1 north east)     coordinate (#1 south east)
(#1 center     -| #1 south west)     coordinate (#1 west)
(#1 center     -| #1 north east)     coordinate (#1 east)
}}}

\usepackage{listings}
\usepackage{multimedia}
\usepackage{graphicx}

\usepackage{colortbl}
\definecolor{cA}{RGB}{212, 237, 244}%
\definecolor{cAa}{RGB}{192, 217, 224}%
\definecolor{cB}{RGB}{237, 212, 244}%
\definecolor{cC}{RGB}{244, 237, 212}%
\definecolor{darkgreen}{RGB}{0, 170, 0}
\newcommand\colorA{\cellcolor{cA}}
\newcommand\colorAa{\cellcolor{cAa}}
\newcommand\colorB{\cellcolor{cB}}
\newcommand\colorC{\cellcolor{cC}}
\newcommand{\ai}{{\color{darkgreen}a_i}}
\newcommand{\Twoai}{{\color{darkgreen}2a_i}}
\newcommand{\bi}{{\color{orange}b_i}}

\DeclareMathOperator{\syz}{Syz}
\DeclareMathOperator{\lt}{LT}
\DeclareMathOperator{\sat}{Sat}

% Matlab code
\usepackage[numbered,framed]{matlab-prettifier}
\lstset{
style              = Matlab-editor,
%  basicstyle         = \mlttfamily,
%  escapechar         = ",
%  mlshowsectionrules = true,
}

\usepackage{tikz-3dplot}
\usetikzlibrary{calc,arrows.meta,positioning,backgrounds}
\pgfdeclarelayer{foreground}  % Used for correct opacity
\pgfdeclarelayer{background}  % Used for correct opacity
\pgfsetlayers{background,main,foreground}

\usetheme{metropolis}           % Use metropolis theme
\title{\vspace{0.3cm}{\small Practical aspects of solving a system of polynomial equations}\vspace{-0.5cm}}
\author{\vspace{-0.3cm}Marcus Valtonen Örnhag}
\institute{%
    Centre for Mathematical Sciences, Lund University}
\date{\vspace{0.2cm}{\footnotesize\emph{Workshop in Algebraic Geometry, May 2021}}}
\titlegraphic{%
    \includegraphics[height=1.7cm]{LundUniversity_C2line_CMYK}
}

\begin{document}
\maketitle

\section{Introduction}

\begin{frame}{Introduction}
Today's tutorial/workshop will contain:
\begin{itemize}
\item A brief theoretical walkthrough,
\item Tips and tricks on how to practically speed-up your solver,
\item Hands-on experience with the automatic Gröbner basis generator,
\item A group exercise followed by a discussion session.
\end{itemize}
\end{frame}

\section{Theory}
\begin{frame}{Recap: Polynomial systems of equations}
Using multi-index notation,
let~$\mat{x^{\alpha}}$ denote a \emph{monomial} of degree $|\mat{\alpha}|$.
Our main goal is to solve a polynomial system of equations
\begin{equation}\label{kappa:eq:polsys}
    \begin{aligned}
    f_1(\mat{x}) & = 0, \\
                 & \hspace{0.575em} \vdots \\
    f_s(\mat{x}) & = 0,
    \end{aligned}
\end{equation}
where each polynomial equation can be expressed
as~$f=\sum_{\mat{\alpha}}c_{\mat{\alpha}}\mat{x^\alpha}$.

\end{frame}

\begin{frame}{Recap: The toolbox}
\begin{itemize}
\item  The set of all solutions to~\eqref{kappa:eq:polsys} is called an \emph{affine variety}, and is denoted
$\mat{V}(f_1,\ldots,f_s)\subset\C$.
\item An \emph{ideal} $I\subset\C[\mat{x}]$ is an additive group satisfying the \emph{absorption property},~\ie{}~if~$f\in I$ and $h\in\C[\mat{x}]$ then $hf\in I$. Moreover, given a set of polynomials $f_1,\ldots,f_s\in\C[\mat{x}]$
we consider the ideal
\begin{equation*}
\langle f_1,\ldots,f_s\rangle = \left\{\sum_{i=1}^sh_if_i\;:\;h_1,\ldots,h_s\in\C[\mat{x}]\right\},
\end{equation*}
which we will refer to as the \emph{ideal generated by} $f_1,\ldots,f_s$.

\end{itemize}
Here~$\C[\mat{x}]$ denotes the set
of polynomials in~$\mat{x}$ with coefficients in $\C$.
\end{frame}


\begin{frame}{Recap: The toolbox (cont.)}

\begin{itemize}
\item The \emph{Hilbert Basis Theorem} states that every ideal of $\C[\mat{x}]$ is
finitely generated, \ie{} given an ideal~$I\subset\C[\mat{x}]$ there exist
$f_1,\ldots,f_s\in\C[\mat{x}]$ such that $I=\langle f_1,\ldots,f_s\rangle$.
\item The polynomial system of equations~\eqref{kappa:eq:polsys} is defined by the
generated ideal; however, the basis is in general not unique.
\end{itemize}

\alert{Solution}: Impose a \emph{monomial ordering} $\longrightarrow$ unique representation.
\end{frame}

\begin{frame}{Recap: The toolbox (cont.)}

\begin{itemize}
\item The \emph{leading term} (\wrt{} the monomial ordering) is well-defined. We denote it~$\lt(\cdot)$.
\item For an ideal $I$ define $\lt(I)\coloneqq\{\lt(f)\;:\;f\in I\}$.
\item A \emph{Gröbner basis} is a finite subset \mbox{$\mat{G}=\{g_1,\ldots,g_t\}\subset I$}
satisfying
\begin{equation*}
    \langle\lt(g_1),\ldots,\lt(g_t)\rangle=\lt(I)\;.
\end{equation*}
\end{itemize}
\end{frame}

\begin{frame}{Recap: Why is all this necessary?}
How does this help us solve the original system?

$\quad\longrightarrow\quad$ We can compute a Gröbner basis efficiently using~\emph{Buchberger's algorithm}.
\end{frame}

\begin{frame}{Recap: The action matric method}
Assume zero-dimensional ideals, \ie{} finitely many solutions to the original system.
\begin{itemize}
\item Define the \emph{coset} $[f] \coloneqq \{f+h\;|\;h\in I\}$.
\item The linear operator $T_f\::\:\C[\mat{x}]/I\longrightarrow\C[\mat{x}]/I$, defined as $T_g([g])=[fg]$,
satisfies $T_f=T_g$ iff $f-g\in I$.
% \item The quotient space $\C[\mat{x}]/I$ is finite dimensional if the original system of
% equations have finitely many solutions, hence
\item The operation $T_\alpha(\cdot)$ can be
represented using a matrix~$\mat{M}_\alpha$. This is the \emph{action matrix}.
\end{itemize}
\end{frame}

\begin{frame}{Recap: The action matric method (cont.)}
Select a monomial basis $\mathcal{B}=\{[\mat{x}^{\beta_i}]\}_{i=1}^K$ for~$\C[\mat{x}] / I$.
Then
\begin{equation*}
    T_\alpha([\mat{x}^{\beta_i}]) = [\alpha(\mat{x})\mat{x}^{\beta_i}] = \sum_{j=1}^Km_{ij}[\mat{x}^{\beta_j}]\;.
\end{equation*}
This implies that, for $\mat{x}\in V(I)$,
\begin{equation*}\label{kappa:eq:action}
    \alpha(\mat{x})\mat{x}^{\beta_i}=\sum_{j=1}^Km_{ij}\mat{x}^{\beta_j}\;.
\end{equation*}
Let $\mat{b}$ represent the basis~$\mathcal{B}$ (as a vector), then we have an \alert{eigenvalue problem}
\begin{equation*}\label{eq:actionmat}
    \alpha(\mat{x})\mat{b}(\mat{x}) = \mat{M}_\alpha\mat{b}(\mat{x})\;.
\end{equation*}
\end{frame}

\begin{frame}{The reality}
For any practical applications in computer vision, robotics and neighboring fields:
\begin{itemize}
\item We need fast or even real-time performance,
\item The input is contaminated with noise. $\longrightarrow$ Several different inputs are
required for robustness.
\end{itemize}

Unfortunately, \emph{Buchberger's algorithm} is NP-complete and often quite heavy. More
efficient algorithms exist, such as Faugère's F4 and F5 algorithms, but they are
still not feasible.
\end{frame}


\begin{frame}{The reality (cont.)}
But there are good news!
We are often interested in solving generic problem instances, \eg{} the relative pose
problem
\begin{equation*}
\begin{aligned}
    \mat{x}'_i\mat{E}\mat{x}_i &= 0, &i=1,\ldots,5\\
    \det(\mat{E}) &= 0, &\\
    2\mat{EE}^\T\mat{E}- \tr(\mat{EE}^\T)\mat{E} &= 0, &
\end{aligned}
\end{equation*}
Only the input data $\mat{x}_i$ and $\mat{x}_i'$ change!
\end{frame}

\begin{frame}{The reality (cont.)}
All non-degenearte instances of the same problem share the same characteristics:
\begin{itemize}
\item The same number of solutions,
\item Gröbner basis the same (\wrt{} a monomial ordering, often GREVLEX)${}^*$,
\item Creating the action matrix involves the same kind of computations${}^*$.
\end{itemize}
${}^*$Only differs for each instance in terms of the coefficients from the input data.
\end{frame}

\begin{frame}{Elimination templates}
\alert{Idea:} We try to find the action matrix using \emph{elimination templates},
which encode the relevant eliminations directly.
\end{frame}

\begin{frame}{Example}
\end{frame}

\begin{frame}{Elimination templates (cont.)}
Consider the $i$:th row in $\alpha\mat{b} = \mat{M}_\alpha\mat{b}$.
Any way of expressing $[\alpha b_i]$ reduces to find a polynomial
\begin{equation}\label{eq:polys}
    p_i = \alpha b_i -\sum_{j=1}^Km_{ij}b_{j}\in I\;.
\end{equation}
\alert{Goal:} Find a \emph{single} polynomial like this expressed in elements from
$\mat{b}$ of $\alpha\mat{b}$.
\end{frame}

\begin{frame}{Elimination templates (cont.)}
\begin{columns}
    \begin{column}{0.5\textwidth}
\alert{Solution:}
Augment the original system
for some monomials~$m_1,\ldots, m_\ell$.
    \end{column}%
    \begin{column}{0.5\textwidth}
\footnotesize
\begin{equation*}\label{kappa:eq:polsys-aug}
    \begin{aligned}
    \left.
    \begin{aligned}
    f_1(\mat{x}) & = 0, \\
                 & \hspace{0.575em} \vdots \\
    f_s(\mat{x}) & = 0, \\
    \end{aligned}
    \right\}&\text{original}
    \\
    \left.
    \begin{aligned}
    m_1f_1(\mat{x}) & = 0, \\
                 & \hspace{0.575em} \vdots \\
    m_1f_s(\mat{x}) & = 0, \\
                 & \hspace{0.575em} \vdots \\
    m_\ell f_1(\mat{x}) & = 0, \\
                 & \hspace{0.575em} \vdots \\
    m_\ell f_s(\mat{x}) & = 0,
    \end{aligned}
    \right\}&\text{augmented}
    \end{aligned}
\end{equation*}
    \end{column}
\end{columns}
\end{frame}

\begin{frame}{Elimination templates (cont.)}
Let the augmented system be represented as~$\mat{CX}=0$ and partition it as follows
\begin{equation*}
    \begin{bmatrix}
        \CE &
        \CR &
        \CB
    \end{bmatrix}
    \begin{bmatrix}
        \XE \\
        \XR \\
        \XB
    \end{bmatrix}
    = 0,
\end{equation*}
where
\begin{itemize}
\item $\XB$ are the basis monomials,
\item $\XR$ are the \emph{reducible} monomials,
\item $\XE$~\emph{excessive} monomials.
\end{itemize}
\end{frame}

\begin{frame}{Elimination templates (cont.)}
We don't want the excessive monomials $\longrightarrow$ eliminate them
\begin{equation*}
    \begin{bmatrix}
        \mat{U}_\mathcal{E}' &
        \CRk{1}' &
        \CBk{1}' \\
        \mat{0} &
        \CRk{2}' &
        \CBk{2}'
    \end{bmatrix}
    \begin{bmatrix}
        \XE \\
        \XR \\
        \XB
    \end{bmatrix}
    = 0\;.
\end{equation*}
We may now discard the first row.
\end{frame}

\begin{frame}{Elimination templates (cont.)}
Assuming $\CRk{2}'$ is of full rank
\begin{equation*}
    \begin{bmatrix}
        \mat{I} &
        \CRk{2}'^{-1}\CBk{2}'
    \end{bmatrix}
    \begin{bmatrix}
        \XR \\
        \XB
    \end{bmatrix}
    = 0,
\end{equation*}
or
\begin{equation*}
    \XR = -\CRk{2}'^{-1}\CBk{2}'\XB\;.
\end{equation*}
Now, the reducible monomials can be expressed as a linear combination of the basis elements,
which is what we were looking for.
\end{frame}

\begin{frame}{Example}
\end{frame}

\begin{frame}{Elimination templates (cont.)}
\begin{itemize}
\item Which monomials should I use to expand the system?
    \begin{itemize}
    \item The fewer the better---fewer operations and better numerics.
    \end{itemize}
\item How to pick these monomials optimally?
    \begin{itemize}
    \item Computationally intensive.
    \end{itemize}
\end{itemize}
\end{frame}

\begin{frame}{Syzygy-based reduction}
\begin{definition}
Let~$(f_1,\ldots,f_t)$ be an ordered $t$-tuple of elements
$f_i\in\K[x]$, then the \emph{(first) syzygy module} is
\begin{equation*}
\syz(f_1,\ldots,f_t) \coloneqq \left\{
    (s_1,\ldots,s_t)\;|\; \sum_{i=1}^t s_if_i=0, \;\mathrm{ and }\; s_i\in\K[x]
\right\}\;.
\end{equation*}
\end{definition}
The syzygy module encodes the ambiguity of polynomials in an ideal.
Let $p\in \langle f_1,\ldots f_t\rangle$, then
\begin{equation*}
    p = \sum_{i=1}^t h_if_i = \sum_{i=1}^t (h_i+s_i)f_i, \qquad (s_1,\ldots,s_t)\in\syz(f_1,\ldots,f_t)\;.
\end{equation*}
\end{frame}

\begin{frame}{Example}
See Viktor Larsson's PhD thesis Example 1.1 (p. 46--47).
\end{frame}

\begin{frame}{Saturation of an ideal}
\begin{definition}
The \emph{saturation} of an ideal~$I\subset\K[x]$ \wrt{} the polynomial $f_s\in\K[x]$ is
defined as
\begin{equation*}
    \sat(I,\, f_s) \coloneqq \{p\;|\;\exists N\geq 0,\, f_s^N p\in I  \}\;.
\end{equation*}
\end{definition}
\alert{Interpretation:} we can remove solutions for which~$f_s(x) =0$.
\end{frame}

\section{Applications}
\begin{frame}{The importance of parameterization}

\end{frame}

\end{document}
